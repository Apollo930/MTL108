\documentclass[12pt]{article}
\usepackage[paper=letterpaper,margin=2cm]{geometry}
\usepackage{amsmath}
\usepackage{amssymb}
\usepackage{amsfonts}
\usepackage{newtxtext, newtxmath}
\usepackage{enumitem}
\usepackage{titling}
\usepackage[colorlinks=true]{hyperref}

\setlength{\droptitle}{-6em}

\title{MTL108 Notes}
\author{Aditya Upadhyay}
\date{}

\begin{document}
\maketitle
\section{Measures of Central Tendency}
\subsection{Arithmetic Mean}
The arithmetic mean of a set of numbers $x_1, x_2 \dots x_n$ is typically denoted using an overhead bar, $\bar {x}$.
\[\bar{x} = \frac{ \sum_{i=1}^{i=n} {x_i}{f_i} }{\sum_{i=1}^{i=n} f_i} \]

\subsection{Geometric Mean}
The geometric mean is defined as the nth root of the product of n numbers, i.e., for a set of numbers $a_1, a_2, \dots a_n$, the geometric mean is defined as
\[ {\left( \prod_{i=1}^{n} a_i \right)}^{1/n} \]

\subsection{Harmonic Mean}
The harmonic mean can be expressed as the reciprocal of the arithmetic mean of the reciprocals of the given set of observations:
\[ H= {\left( \frac{\sum_{i=1}^{n} {x_i}^{-1} }{n}\right)}^{-1}\]

\subsection{Median}
The median of a finite list of numbers is the "middle" number, when those numbers are listed in order from smallest to greatest.
For grouped data, \[ l + \left(\frac{\frac{N}{2}-CF}{f}\right)h\]
where l=lower limit of median class,
    CF= cumulative frequency of the class preceding the median class,
    f= frequency of the median class
    N=total number of 

\subsection{Mode}
\paragraph{text}
Skewness, kurtosis
\newpage
\section{Methods of Dispersion}
\subsection{Standard Deviation}
It is the square root of the expectation of the squared deviation of a random variable from its population mean.
\[ s_N= \sqrt{\frac{1}{N} \sum_{i=1}^{N} {(x_i - \bar{x})}^2 }\]
\subsection{Correlation}

\section{Probability}

\end{document}